%!TEX TS-program = xelatex
%!TEX encoding = UTF-8 Unicode
% Awesome CV LaTeX Template for Cover Letter
%
% This template has been downloaded from:
% https://github.com/posquit0/Awesome-CV
%
% Authors:
% Claud D. Park <posquit0.bj@gmail.com>
% Lars Richter <mail@ayeks.de>
%
% Template license:
% CC BY-SA 4.0 (https://creativecommons.org/licenses/by-sa/4.0/)
%


%-------------------------------------------------------------------------------
% CONFIGURATIONS
%-------------------------------------------------------------------------------
% A4 paper size by default, use 'letterpaper' for US letter
\documentclass[11pt, a4paper]{Awesome-CV/awesome-cv}

\usepackage[ngerman]{babel}

% Configure page margins with geometry
\geometry{left=2.5cm, top=.8cm, right=2.0cm, bottom=1.8cm, footskip=.5cm}

% Specify the location of the included fonts
\fontdir[Awesome-CV/fonts/]

% Color for highlights
\input{scripts/colors}

% Set false if you don't want to highlight section with awesome color
\setbool{acvSectionColorHighlight}{true}

% If you would like to change the social information separator from a pipe (|) to something else
\renewcommand{\acvHeaderSocialSep}{\quad\textbar\quad}


%-------------------------------------------------------------------------------
%	PERSONAL INFORMATION
%	Comment any of the lines below if they are not required
%-------------------------------------------------------------------------------
% Available options: circle|rectangle,edge/noedge,left/right
\photo{images/Bewerbungsfoto.JPG}
\name{Maximilian}{Sackel}
\position{Software Architect{\enskip\cdotp\enskip}Security Expert}
\address{Meckinckweg 15, 44309 Dortmund, Germany}

\mobile{(+49) 1578-3291423}
\email{mail@maxsac.de}
\homepage{maxsac.github.com}
\github{maxsac}
% \gitlab{gitlab-id}
% \stackoverflow{SO-id}{SO-name}
% \twitter{@twit}
% \skype{skype-id}
% \reddit{reddit-id}
% \medium{madium-id}
% \googlescholar{googlescholar-id}{name-to-display}
%% \firstname and \lastname will be used
% \googlescholar{googlescholar-id}{}
% \extrainfo{extra informations}

% \quote{``Be the change that you want to see in the world."}



%-------------------------------------------------------------------------------
%   LETTER INFORMATION
%   All of the below lines must be filled out
%-------------------------------------------------------------------------------
% The company being applied to
\recipient
{REWE Digital}
{Christoph Lexa \\Schanzenstrasse 6-20\\51063 Köln}
% The date on the letter, default is the date of compilation
\letterdate{\today}
% The title of the letter
\lettertitle{Stellenausschreibung zum Big Data Engineer}
% How the letter is opened
\letteropening{\\\\\\Sehr geehrte Herr Lexa,}
% How the letter is closed
\letterclosing{Mit freundlichen Grüßen}
\sign{images/sign.png}
% Any enclosures with the letter
\letterenclosure[Anhang]{Curriculum Vitae, Zeugnisse}


%-------------------------------------------------------------------------------
\begin{document}

% Print the header with above personal informations
% Give optional argument to change alignment(C: center, L: left, R: right)
\makecvheader[R]

% Print the footer with 3 arguments(<left>, <center>, <right>)
% Leave any of these blank if they are not needed
\makecvfooter
{\today}
{Maximilian Sackel~~~·~~~Anschreiben}
{}

% Print the title with above letter informations
\makelettertitle

%-------------------------------------------------------------------------------
%   LETTER CONTENT
%-------------------------------------------------------------------------------
\begin{cvletter}
    die Digitalisierung der Prozesse im Lebensmittelhandel bietet mir
    zahlreiche Möglichkeiten, meine Kompetenzen zur Datenanalyse sowie Monte-Carlo Simulationen in
    verschiedenen Aufgabenbereichen bei Ihnen einzubringen und durch Förderungen weiter auszubauen.
    Die datengetriebene Optimierung von Supply Chain-Prozessen und
    Echtzeitprognossen von Auslastungen sowie Kapazitäten ist ein Themenfeld,
    das sowohl ökologisch als auch ökonomisch von stetig wachsender Bedeutung
    ist.
    Die Implementierung und Betreuung von neuen Data Science Use Cases ist mir
    ein aus der Astrophysik vertrauter Prozess,
    welchen ich gerne auf neue Themenfelder übertragen möchte.

    Das Masterstudium in Physik habe ich an der Technischen Universität Dortmund
    abgeschlossen.
    Schwerpunkte während meines Studiums waren die Themen \textit{``Maschinelles Lernen''} und \textit{``Computational Physics''}.
    Am Lehrstuhl für Astroteilchenphysik arbeitete ich an der Entwicklung von Monte-Carlo-Simulationen.
    Dabei lernte ich das kollaborative Arbeiten an größeren Softwareprojekten und festigte den sicheren Umgang mit \textit{C++} und \textit{Python}.
    Basierend auf den Simulationen werden statistische Big-Data-Analysen für
    Messexperimente durchgeführt, an deren Entwicklung ich aktiv mitarbeitete.
    Dabei werden beispielsweise in Echtzeitanalysen oder großen Datensätzen
    nach ausgezeichneten Ereignissen gesucht und daraus freie Parameter für Modelle vorhergesagt.

    Durch die Konzeption und Durchführung der Übungen im Kurs
    \textit{``Statistische Methoden der Datenanalyse''} sowie Präsentationen
    auf größeren Konferenzen konnte ich erste Erfahrungen sammeln, mein Wissen mit anderen Menschen
zu teilen. In meiner Freizeit betreibe ich leistungsorientiertes Judo, bei dem es um den disziplinierten und fairen Wettkampf sowohl im Team als auch alleine geht. Die Arbeit an kleineren Programmierprojekten mit Freunden bereitet mir
viel Freude. Dabei wird eine praktikable Lösung in einer limitierten Zeit gesucht, um sich mit anderen Teams zu messen.

    Mit Engagement und Interesse an der Umsetzung theoretischer Modelle in anwendungsbezogenen Prozessen möchte ich mich gerne in Ihr Team einbringen.
    Ich freue mich darauf, Sie in einem persönlichen Gespräch zu überzeugen.

\end{cvletter}


%-------------------------------------------------------------------------------
% Print the signature and enclosures with above letter informations
\makeletterclosing

\end{document}
