%!TEX TS-program = xelatex
%!TEX encoding = UTF-8 Unicode
% Awesome CV LaTeX Template for Cover Letter
%
% This template has been downloaded from:
% https://github.com/posquit0/Awesome-CV
%
% Authors:
% Claud D. Park <posquit0.bj@gmail.com>
% Lars Richter <mail@ayeks.de>
%
% Template license:
% CC BY-SA 4.0 (https://creativecommons.org/licenses/by-sa/4.0/)
%


%-------------------------------------------------------------------------------
% CONFIGURATIONS
%-------------------------------------------------------------------------------
% A4 paper size by default, use 'letterpaper' for US letter
\documentclass[11pt, a4paper]{Awesome-CV/awesome-cv}

\usepackage[ngerman]{babel}

% Configure page margins with geometry
\geometry{left=2.5cm, top=.8cm, right=2.0cm, bottom=1.8cm, footskip=.5cm}

% Specify the location of the included fonts
\fontdir[Awesome-CV/fonts/]

% Color for highlights
\input{scripts/colors}

% Set false if you don't want to highlight section with awesome color
\setbool{acvSectionColorHighlight}{true}

% If you would like to change the social information separator from a pipe (|) to something else
\renewcommand{\acvHeaderSocialSep}{\quad\textbar\quad}


%-------------------------------------------------------------------------------
%	PERSONAL INFORMATION
%	Comment any of the lines below if they are not required
%-------------------------------------------------------------------------------
% Available options: circle|rectangle,edge/noedge,left/right
\photo{../images/Bewerbungsfoto.JPG}
\name{Maximilian}{Sackel}
\position{Software Architect{\enskip\cdotp\enskip}Security Expert}
\address{Meckinckweg 15, 44309 Dortmund, Germany}

\mobile{(+49) 1578-3291423}
\email{mail@maxsac.de}
\homepage{maxsac.github.com}
\github{maxsac}
% \gitlab{gitlab-id}
% \stackoverflow{SO-id}{SO-name}
% \twitter{@twit}
% \skype{skype-id}
% \reddit{reddit-id}
% \medium{madium-id}
% \googlescholar{googlescholar-id}{name-to-display}
%% \firstname and \lastname will be used
% \googlescholar{googlescholar-id}{}
% \extrainfo{extra informations}

% \quote{``Be the change that you want to see in the world."}



%-------------------------------------------------------------------------------
%   LETTER INFORMATION
%   All of the below lines must be filled out
%-------------------------------------------------------------------------------
% The company being applied to
\recipient
{Volkswagen Infotainment}
{Universitätsstr. 140\\44799 Bochum\\Deutschland}
% The date on the letter, default is the date of compilation
\letterdate{\today}
% The title of the letter
\lettertitle{Stellenausschreibung zum Analyst Open Source Software}
% How the letter is opened
\letteropening{\\\\\\Sehr geehrte Frau Thülig,}
% How the letter is closed
\letterclosing{Mit freundlichen Grüßen}
\sign{images/sign.png}
% Any enclosures with the letter
\letterenclosure[Anhang]{Curriculum Vitae, Zeugnisse}


%-------------------------------------------------------------------------------
\begin{document}

% Print the header with above personal informations
% Give optional argument to change alignment(C: center, L: left, R: right)
\makecvheader[R]

% Print the footer with 3 arguments(<left>, <center>, <right>)
% Leave any of these blank if they are not needed
\makecvfooter
{\today}
{Maximilian Sackel~~~·~~~Anschreiben}
{}

% Print the title with above letter informations
\makelettertitle

%-------------------------------------------------------------------------------
%   LETTER CONTENT
%-------------------------------------------------------------------------------
\begin{cvletter}
    das Unternehmen Volkswagen als einer der führenden Automobilhersteller bietet mir aufgrund seiner großen Bandbreite an innovativen Lösungen viele Möglichkeiten, meine Kompetenzen zur Datenanalyse und Monte-Carlo Simulationen einzubringen.
    Bei der Entwicklung neuer Software und deren Wartung auf Open Source Software (OSS) zu setzen, ist mir eine große Herzensangelegenheit, bei der sowohl die Seite des Unternehmens als auch die OSS-Community sehr profitieren kann.
    Den Weg hin zur nachhaltigen Mobilität einerseits ökologisch als auch softwaretechnisch möchte ich mit Ihnen zusammen bestreiten.
    % Redundante Arbeit kann durch die Verwendung vermieden und eine größere Testumfang der Software erreicht werden, was bei stark digitalisierten Fahrzeugen eine stetig wachsende Herrausforderung darstellt.


    Im Rahmen meines Masterstudiums in Physik an der Technischen Universität Dortmund habe ich bereits zahlreiche Erfahrungen im alltäglichen Umgang mit OSS sammeln können.
    Schwerpunkte während meines Studiums waren die Themen \textit{``Maschinelles Lernen''} und \textit{``Computational Physics''}.
    Am Lehrstuhl für Astroteilchenphysik arbeitete ich an der Entwicklung von Monte-Carlo-Simulationen von leptonischen Wechselwirkungen in Teilchenschauern.
    Teil der Aufgabe war es, mehrere OSS-Projekte zu verbinden, um selbstentwickelten Code durch performanteren und besser validierten zu ersetzen.
    Basierend auf den Simulationen werden statistische Big-Data-Analysen für Messexperimente durchgeführt, bei denen ich mich aktiv an der Entwicklung beteiligen konnte.

    Durch die Erstellung und Durchführung der Übungen im Kurs \textit{``Statistische Methoden der Datenanalyse''} sowie Präsentationen auf größeren Konferenzen konnte ich erste Erfahrungen sammeln, mein Wissen mit anderen Menschen zu teilen.
    In meiner Freizeit betreibe ich leistungsorientiertes Judo, bei dem es um den disziplinierten und fairen Wettkampf sowohl im Team als auch alleine geht.
    Des Weiteren bringe ich meine Expertise bezüglich OSS und Open Data in einer demokratischen Partei ein und versuche Parteifreunde von den Konzepten zu überzeugen.

Mit viel Engagement und Interesse würde ich gerne das Potenzial sowie die rechtlichen Rahmenbedingungen von OSS bestimmen und mich in Ihr Team einbringen.
Ich freue mich darauf, Sie in einem persönlichen Gespräch zu überzeugen.

\end{cvletter}


%-------------------------------------------------------------------------------
% Print the signature and enclosures with above letter informations
\makeletterclosing

\end{document}
