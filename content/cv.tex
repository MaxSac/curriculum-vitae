%!TEX TS-program = xelatex
%!TEX encoding = UTF-8 Unicode
% Awesome CV LaTeX Template for CV/Resume
%
% This template has been downloaded from:
% https://github.com/posquit0/Awesome-CV
%
% Author:
% Claud D. Park <posquit0.bj@gmail.com>
% http://www.posquit0.com
%
% Template license:
% CC BY-SA 4.0 (https://creativecommons.org/licenses/by-sa/4.0/)
%


%-------------------------------------------------------------------------------
% CONFIGURATIONS
%-------------------------------------------------------------------------------
% A4 paper size by default, use 'letterpaper' for US letter
\documentclass[11pt, a4paper]{Awesome-CV/awesome-cv}

% Configure page margins with geometry
\geometry{left=1.4cm, top=.8cm, right=1.4cm, bottom=1.8cm, footskip=.5cm}

% Specify the location of the included fonts
\fontdir[fonts/]

% Color for highlights
\input{scripts/colors}

% Set false if you don't want to highlight section with awesome color
\setbool{acvSectionColorHighlight}{true}

% If you would like to change the social information separator from a pipe (|) to something else
\renewcommand{\acvHeaderSocialSep}{\quad\textbar\quad}


%-------------------------------------------------------------------------------
%	PERSONAL INFORMATION
%	Comment any of the lines below if they are not required
%-------------------------------------------------------------------------------
% Available options: circle|rectangle,edge/noedge,left/right
\photo{images/Bewerbungsfoto.JPG}
\name{Maximilian}{Sackel}
\position{Software Architect{\enskip\cdotp\enskip}Security Expert}
\address{Meckinckweg 15, 44309 Dortmund, Germany}

\mobile{(+49) 1578-3291423}
\email{mail@maxsac.de}
\homepage{maxsac.github.com}
\github{maxsac}
% \gitlab{gitlab-id}
% \stackoverflow{SO-id}{SO-name}
% \twitter{@twit}
% \skype{skype-id}
% \reddit{reddit-id}
% \medium{madium-id}
% \googlescholar{googlescholar-id}{name-to-display}
%% \firstname and \lastname will be used
% \googlescholar{googlescholar-id}{}
% \extrainfo{extra informations}

% \quote{``Be the change that you want to see in the world."}


%-------------------------------------------------------------------------------
\begin{document}

% Print the header with above personal informations
% Give optional argument to change alignment(C: center, L: left, R: right)
\makecvheader

% Print the footer with 3 arguments(<left>, <center>, <right>)
% Leave any of these blank if they are not needed
\makecvfooter
{\today}
{Maximilian Sackel~~~·~~~Curriculum Vitae}
{\thepage}


%-------------------------------------------------------------------------------
%   CV/RESUME CONTENT
%   Each section is imported separately, open each file in turn to modify content
%-------------------------------------------------------------------------------

\cvsection{Education}
\begin{cventries}
    \cventry
    {M.Sc. in Physics} % Degree
    {TU Dortmund (Technical University Dortmund)} % Institution
    {Dortmund, Germany} % Location
    {Oct. 2018 - Jun. 2021} % Date(s)
    {
        \begin{cvitems} % Description(s) bullet points
        \item {Thesis: First electromagnetic interaction modelin CORSIKA8 using the Monte Carlosimulation tool PROPOSAL.}
        \end{cvitems}
    }%

    \cventry
    {B.Sc. in Physics} % Degree
    {TU Dortmund (Technical University Dortmund)} % Institution
    {Dortmund, Germany} % Location
    {Oct. 2014 - Sep. 2018} % Date(s)
    {
        \begin{cvitems} % Description(s) bullet points
        \item {Thesis: Gamma/Hadron separation with measured background data with FACT.}
        \end{cvitems}
    }%

    \cventry
    {Hochschulreife} % Degree
    {KKG Dortmund (Käthe-Kollwitz-Gymnasium Dortmund)} % Institution
    {Dortmund, Germany} % Location
    {Aug. 2006 - Jun. 2014} % Date(s)
    {
        \begin{cvitems} % Description(s) bullet points
        \item {Advanced courses in mathematics and physics.}
        \end{cvitems}
    }%
\end{cventries}

\cvsection{Skills}
\begin{cvskills}

    %%---------------------------------------------------------
    %\cvskill
    %{DevOps} % Category
    %{AWS, Docker, Kubernetes, Rancher, Vagrant, Packer, Terraform, Jenkins, CircleCI} % Skills

    %%---------------------------------------------------------
    %\cvskill
    %{Back-end} % Category
    %{Koa, Express, Django, REST API} % Skills

    %---------------------------------------------------------
    \cvskill
    {Build Automation} % Category
    {CMake, Make, Conan, pip} % Skills

    \cvskill
    {3D Printing}
    {Onshape, OpenSCAD, PrusaSlicer, Octoprint}

    %%---------------------------------------------------------
    %\cvskill
    %{Operating Systems} % Category
    %{Linux} % Skills

    %---------------------------------------------------------
    \cvskill
    {Programming} % Category
    {C++17, Python3, LaTeX, SQLite, git} % Skills

    %---------------------------------------------------------
    \cvskill
    {ML \& Data Mining} % Category
    {pandas, scipy, scikit-learn, PyTorch} % Skills

    %---------------------------------------------------------
    \cvskill
    {Languages} % Category
    {German, English} % Skills

    %---------------------------------------------------------
\end{cvskills}

\cvsection{Experience}
\begin{cventries}

    %---------------------------------------------------------
    \cventry
    {Teaching Assistant} % Job title
    {Statistical methods of data analysis (Prof.~Dr.~Dr.~W.~Rhode)} % Organization
    {Dortmund, Germany} % Location
    {Oct. 2018 - Sep. 2020} % Date(s)
    {
        \begin{cvitems} % Description(s) of tasks/responsibilities
        % \item {The appropriate use of statistical methods for the analysis of
        %     moderate to very large amounts of data is instucted on the basis of the chronological sequence of a data analysis.}
        \item {Teaching 4/5-B.Sc.\ or 1/2-M.Sc.~students in the appropriate use of statistical methods to analyze moderate to very large data sets.}
        \end{cvitems}
    }

    %---------------------------------------------------------
    \cventry
    {Working group Member} % Job title
    {Towards a next generation of CORSIKA} % Organization
    {Karlsruhe, Germany} % Location
    {Nov. 2019 - PRESENT} % Date(s)
    {
        \begin{cvitems} % Description(s) of tasks/responsibilities
        \item {Experiences in working group organized development of the most used simulation program in astroparticle physics.}
        \end{cvitems}
    }

\end{cventries}

%-------------------------------------------------------------------------------
%   SECTION TITLE
%-------------------------------------------------------------------------------
\cvsection{Presentation}


%-------------------------------------------------------------------------------
%   CONTENT
%-------------------------------------------------------------------------------
\begin{cventries}

    %---------------------------------------------------------
    \cventry
    {Presenter for <Electromagnetic shower simulation using PROPOSAL>} % Role
    {CORSIKA Cosmic Ray Simulation Workshop at Karlsruhe Institute of Technology} % Event
    {Karlsruhe, Germany} % Location
    {Jun. 2020} % Date(s)
    {
        \begin{cvitems} % Description(s)
        \item {Introduce the first leptonic and photonic interaction model in
                the recent version of the Monte Carlo library for extensiv air
            shower.}
        \item {Insights into how the underlying runtime intensive computations can be performed effectively on resource limited systems.}
        \end{cvitems}
    }

\end{cventries}
\cvsection{Writing}


%-------------------------------------------------------------------------------
%   CONTENT
%-------------------------------------------------------------------------------
\begin{cventries}

    %---------------------------------------------------------
    \cventry
    {Co-Writer} % Role
    {Evaluate electromagnetic modell against state of the art competitors} % Title
    {ICRC Berlin, Germany} % Location
    {Jul. 2021} % Date(s)
    {
        \begin{cvitems} % Description(s)
        \item {Comparisons of typical electromagnetic quantities against standard tools to evaluate the new electromagnetic module based on PROPOSAL.  DOI:~\href{https://doi.org/10.22323/1.395.0428}{10.22323/1.395.0428}
            }
        \end{cvitems}
    }

    %---------------------------------------------------------
    \cventry
    {Co-Writer} % Role
    {Physical extentions of PROPOSAL to support new application areas} % Title
    {ICPPA Moscow, Russian Fed.} % Location
    {Dec. 2020} % Date(s)
    {
        \begin{cvitems} % Description(s)
        \item {Implementation of new photonic and electrical processes for the
                development of an electromagnetic module that can form the basis of
            large-scale high-energy Monte Carlo simulations. DOI:~\href{https://doi.org/10.1088/1742-6596/1690/1/012021}{10.1088/1742-6596/1690/1/012021}}
        \end{cvitems}
    }

    %---------------------------------------------------------
\end{cventries}


\end{document}
