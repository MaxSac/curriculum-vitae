%!TEX TS-program = xelatex
%!TEX encoding = UTF-8 Unicode
% Awesome CV LaTeX Template for CV/Resume
%
% This template has been downloaded from:
% https://github.com/posquit0/Awesome-CV
%
% Author:
% Claud D. Park <posquit0.bj@gmail.com>
% http://www.posquit0.com
%
% Template license:
% CC BY-SA 4.0 (https://creativecommons.org/licenses/by-sa/4.0/)
%


%-------------------------------------------------------------------------------
% CONFIGURATIONS
%-------------------------------------------------------------------------------
% A4 paper size by default, use 'letterpaper' for US letter
\documentclass[11pt, a4paper]{Awesome-CV/awesome-cv}

\usepackage[ngerman]{babel}

% Configure page margins with geometry
\geometry{left=2.5cm, top=.8cm, right=2.0cm, bottom=1.8cm, footskip=.5cm}

% Specify the location of the included fonts
\fontdir[fonts/]

% Color for highlights
\input{scripts/colors}

% Set false if you don't want to highlight section with awesome color
\setbool{acvSectionColorHighlight}{true}

% If you would like to change the social information separator from a pipe (|) to something else
\renewcommand{\acvHeaderSocialSep}{\quad\textbar\quad}


\newcommand{\hobby}[3]{\hbox{{\Large{#1}}\quad\parbox{5.0cm}{\entrytitlestyle{#2}\\[-0.2em]{\descriptionstyle{#3}}}}}%

% Available options: circle|rectangle,edge/noedge,left/right
\photo[circle,noedge,left]{images/Bewerbungsfoto.JPG}

%-------------------------------------------------------------------------------
%	PERSONAL INFORMATION
%	Comment any of the lines below if they are not required
%-------------------------------------------------------------------------------
% Available options: circle|rectangle,edge/noedge,left/right
\photo{images/Bewerbungsfoto.JPG}
\name{Maximilian}{Sackel}
\position{Software Architect{\enskip\cdotp\enskip}Security Expert}
\address{Meckinckweg 15, 44309 Dortmund, Germany}

\mobile{(+49) 1578-3291423}
\email{mail@maxsac.de}
\homepage{maxsac.github.com}
\github{maxsac}
% \gitlab{gitlab-id}
% \stackoverflow{SO-id}{SO-name}
% \twitter{@twit}
% \skype{skype-id}
% \reddit{reddit-id}
% \medium{madium-id}
% \googlescholar{googlescholar-id}{name-to-display}
%% \firstname and \lastname will be used
% \googlescholar{googlescholar-id}{}
% \extrainfo{extra informations}

% \quote{``Be the change that you want to see in the world."}


%-------------------------------------------------------------------------------
\begin{document}

% Print the header with above personal informations
% Give optional argument to change alignment(C: center, L: left, R: right)
\makecvheader

% Print the footer with 3 arguments(<left>, <center>, <right>)
% Leave any of these blank if they are not needed
\makecvfooter
{\today}
{Maximilian Sackel~~~·~~~Curriculum Vitae}
{\thepage}



%-------------------------------------------------------------------------------
%   CV/RESUME CONTENT
%   Each section is imported separately, open each file in turn to modify content
%-------------------------------------------------------------------------------

\cvsection{Bildung und Qualifikationen}
\begin{cventries}
    \cventry
    {M.Sc. in Physik \quad (Abschlussnote: 1.8)} % Degree
    {Technische Universtität Dortmund} % Institution
    {Dortmund, Deutschland} % Location
    {Okt. 2018 - Jun. 2021} % Date(s)
    {
        \begin{cvitems} % Description(s) bullet points
        \item {Masterarbeit: First electromagnetic interaction model in CORSIKA~8 using the Monte Carlo simulation tool PROPOSAL}
        \end{cvitems}
    }%

    \cventry
    {B.Sc. in Physik \quad (Abschlussnote: 2.4)} % Degree
    {Technische Universtität Dortmund} % Institution
    {Dortmund, Deutschland} % Location
    {Okt. 2014 - Sep. 2018} % Date(s)
    {
        \begin{cvitems} % Description(s) bullet points
        \item {Bachelorarbeit: Gamma/Hadron separation with measured background data with FACT}
        \end{cvitems}
    }%

    \cventry
    {Allgemeine Hochschulreife} % Degree
    {Käthe-Kollwitz-Gymnasium} % Institution
    {Dortmund, Deutschland} % Location
    {Aug. 2006 - Jun. 2014} % Date(s)
    {
        \begin{cvitems} % Description(s) bullet points
        \item {Leistungskurse in Mathematik and Physik}
        \end{cvitems}
    }%
\end{cventries}

\cvsection{Kompetenzen}
\begin{cvskills}
    %---------------------------------------------------------
    \cvskill
    {Build Automation} % Category
    {CMake, Make, Conan, pip} % Skills

    % \cvskill
    % {3D Printing}
    % {Onshape, OpenSCAD, PrusaSlicer, Octoprint}

    %---------------------------------------------------------
    \cvskill
    {Programming} % Category
    {C++17, Python3, LaTeX, git} % Skills

    %---------------------------------------------------------
    \cvskill
    {ML \& Data Mining} % Category
    {pandas, scipy, scikit-learn, PyTorch} % Skills

    %---------------------------------------------------------
    \cvskill
    {Sprachen} % Category
    {Deutsch, Englisch} % Skills
\end{cvskills}

\cvsection{Berufserfahrung}
\begin{cventries}
    %---------------------------------------------------------
    \cventry
    {Lehrassistent} % Job title
    {Statistische Methoden der Datenanalyse (Prof.~Dr.~Dr.~W.~Rhode)} % Organization
    {Dortmund, Deutschland} % Location
    {Okt. 2018 - Sep. 2020} % Date(s)
    {
        \begin{cvitems} % Description(s) of tasks/responsibilities
            % \item {The appropriate use of statistical methods for the analysis of
            %     moderate to very large amounts of data is instucted on the basis of the chronological sequence of a data analysis.}
        \item {Unterricht in der sachgerechten Anwendung von statistischen
            Methoden zur Analyse von größeren Datensätzen}
        \end{cvitems}
    }

    %---------------------------------------------------------
    \cventry
    {Arbeitsgruppenmitglied} % Job title
    {Towards a next generation of CORSIKA} % Organization
    {Karlsruhe, Deutschland} % Location
    {Nov. 2019 - GEGENWART} % Date(s)
    {
        \begin{cvitems} % Description(s) of tasks/responsibilities
        \item {Erfahrungen zur organisierten Entwicklung des meistgenutzten
            Simulationsprogamm in der Astroteilchenphysik}
        \end{cvitems}
    }
\end{cventries}

%-------------------------------------------------------------------------------
%   SECTION TITLE
%-------------------------------------------------------------------------------
\cvsection{Publikationen \& Präsentationen}
%-------------------------------------------------------------------------------
%   CONTENT
%-------------------------------------------------------------------------------
\begin{cventries}
    %---------------------------------------------------------
    \cventry
    {Vortragender <Electromagnetic shower simulation using PROPOSAL>} % Role
    {CORSIKA Cosmic Ray Simulation Workshop am KIT} % Event
    {Karlsruhe, Deutschland} % Location
    {Jun. 2020} % Date(s)
    {
        \begin{cvitems} % Description(s)
        \item {Einführung des ersten leptonischen und photonischen
            Wechselwirkungsmodells in der aktuellen Version der
        Monte-Carlo Bibliothek für ausgedehnte Luftschauer}
        \item {Einblicke in die zugrunde liegenden laufzeitintensiven Berechnungen auf ressourcenbeschränkten Systemen}
        \end{cvitems}
    }

    %---------------------------------------------------------
    \cventry
    {Co-Autor} % Role
    {Electromagnetic Shower Simulation for CORSIKA 8} % Title
    {ICRC Berlin, Deutschland} % Location
    {Jul. 2021} % Date(s)
    {
        \begin{cvitems} % Description(s)
        \item {Vergleich typischer elektromagnetischer Größen mit
                Standardwerkzeugen zur Bewertung des neuen elektromagnetischen
                Moduls auf der Grundlage von PROPOSAL
                (DOI:~\href{https://doi.org/10.22323/1.395.0428}{10.22323/1.395.0428})
            }
        \end{cvitems}
    }

    %---------------------------------------------------------
    \cventry
    {Co-Autor} % Role
    {PROPOSAL:\@ A library to propagate leptons and high energy photons} % Title
    {ICPPA Moscow, Russland} % Location
    {Dec. 2020} % Date(s)
    {
        \begin{cvitems} % Description(s)
        \item {Implementierung neuer photonischer und elektrischer Prozesse für
            die Entwicklung eines elektromagnetischen Moduls, das die Grundlage
        für groß angelegte Hochenergie-Monte-Carlo-Simulationen bilden kann (DOI:~\href{https://doi.org/10.1088/1742-6596/1690/1/012021}{10.1088/1742-6596/1690/1/012021})}
        \end{cvitems}
    }
\end{cventries}

%-------------------------------------------------------------------------------
%   SECTION TITLE
%-------------------------------------------------------------------------------
\cvsection{Hobbys}
%-------------------------------------------------------------------------------
%   CONTENT
%-------------------------------------------------------------------------------
\vspace{0.8em}

\hbox{%
    \hobby{\faIcon{running}}{Judo}{zweite Bundesliga}%
    \hfill%
    \hobby{\faIcon[regular]{comments}}{Politik}{Lokales und Digitales}%
    \hfill%
    \hobby{\faIcon{code}}{Coding}{kleinere Hackathons}%
}
\end{document}
